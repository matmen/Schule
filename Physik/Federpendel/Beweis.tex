\documentclass[12pt, a4paper]{article}
\begin{document}
\textbf{Beweise: die Periodendauer T eines Federpendels = $2\pi*\sqrt{\frac{m}{d}}$}\\
\\
$F_G = M*a$\\
$F_{Feder}=-D*s$\\
\\
\\
\textbf{Kr\"aftegleichgewicht:}\\\\
$F_{Feder}=F_g$\\
$-D*s=m*a=m*s''$\\
$-D*s(t)=m*a(t)=m*s''$\\
\\
$s(t) = \hat{s}*sin(\omega * t)$\\
$v(t) = s'(t) = \omega * \hat{s} * cos(\omega * t)$\\
$a(t) = s''(t) = -\omega^2 * \hat{s}*sin(\omega * t)$\\
\\
$-\omega^2 * \hat{s} * sin(\omega * t) * m = -D * \hat{s} * sin(\omega * t)$\\
$-\omega^2*m = -D$\\
$\omega^2 = \frac{D}{m}$\\
$\omega = 2\pi f = \frac{2\pi}{T}$\\
$\frac{2\pi}{T}=\sqrt{\frac{D}{m}}$\\
\\
$T=2\pi*\sqrt{\frac{m}{D}}$ q.e.d.

\end{document}